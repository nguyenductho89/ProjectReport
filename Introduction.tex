\section{Introduction}
\subsection{Introduce to clustering}

\forceindent Cluster analysis or clustering is a very important technology in Data Mining. It divides the datasets into clusters which are collections of data objects with common characteristics based on the computation of data information.

Clustering problems come up in many different applications consisting data mining and knowledge discovery, data compression and vector quantization \cite{Gersho-vector}, and pattern recognition and pattern classification \cite{Kanugo-efficient}. There are several commonly used clustering algorithms, such as K-means, CLARANS \cite{Raymon-clarans}, STING \cite{Wang-sting}, CLIQUE \cite{Rakesh-Automatic}, and CURE \cite{Guha-cure}.


\subsection{Why does clustering be used?}

\subsection{How does clustering work?}

\subsection{Clustering categories}
Clustering algorithms may be classified as listed below:\\
%http://home.deib.polimi.it/matteucc/Clustering/tutorial_html/index.html
+) Exclusive Clustering \\
Data are grouped in an exclusive way, so that if a certain datum belongs to a definite cluster then it could not be included in another cluster. In this report, we will discuss K-mean and Mean Shift algorithm which both are Exclusive Clustering.\\
+) Overlapping Clustering \\
On the contrary the second type, the overlapping clustering, uses fuzzy sets to cluster data, so that each point may belong to two or more clusters with different degrees of membership.In this case, data will be associated to an appropriate membership value.\\
 +) Hierarchical Clustering \\
A hierarchical clustering algorithm is based on the union between the two nearest clusters. The beginning condition is realized by setting every datum as a cluster. After a few iterations it reaches the final clusters wanted.\\\
 +) Probabilistic Clustering\\
This one use a completely probabilistic approach.
