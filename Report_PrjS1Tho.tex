\documentclass[13pt,a4paper]{report}
\usepackage[utf8]{inputenc}
\usepackage{enumerate}
\usepackage{tikz}
\usepackage{pgfplots}

%\usepackage{enumitem}
\usepackage{listings}

\renewcommand{\thesection}{\arabic{section}}
\renewcommand\theparagraph{\thesubsubsubsection.\arabic{paragraph}} % optional; useful if paragraphs are to be numbered


\begin{document}
	
\begin{titlepage}
	\centering
	\vspace{2cm}
	{\huge\bfseries Report of Project of Semester S1: Nonparametric method for image analysis \par}
	\vspace{2cm}
	{\Large\itshape Students:\\
		Loc Thi Thuy Linh\\
		Nguyen Duc Tho\par}
	\vfill
	supervised by\par
	\large Nghiem Thi Phuong \& Tran Giang Son\\
	ICT Department, USTH\par
	\vfill
% Bottom of the page
	{\large \today\par}
\end{titlepage}

\tableofcontents{}
\newpage

\section{Introduction}

\subsection{What is clustering?}

\subsection{Why is clustering used?}

\subsection{How does clustering work?}

\subsection{Clustering categories}
\begin{enumerate}[a.]
	\item Parametric
	\item Non-parametric
\end{enumerate}

\section{K-means}

\subsection{What?}

\subsection{Why?}

\subsection{How?}

\subsection{Advantages and disadvantages}
%lead to the next part


\section{Non-parametric: Meanshift}

\subsection{What?}
Mean shift is a non-parametric feature-space analysis technique for locating the maxima of a density function, a so-called mode-seeking algorithm\textsuperscript{2.1}

\begin{tikzpicture}
\pgfplotsset{compat=1.11}
\begin{axis}[axis lines=middle,xlabel=$x$,ylabel=$y$,enlargelimits]
\end{axis}
\end{tikzpicture}
\subsection{Why?}

\subsection{How?}

\subsection{Advantages and disadvantages}

\section{Evaluation}

\begin{enumerate}[(a) ]
\item What is evaluation? Convergence(performance) | accuracy | Complexity
\item Dataset (name, source, type, image size ....)
\item Computer test configuration
\item Program language
\item Results (time | number of cluster | accuracy) ~~~> describe detail.
\item Comparison of the two methods
\end{enumerate}
	
	
\section{Conclusion}
\section{Reference}

\end{document}