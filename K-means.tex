\section{K-means}
\subsection{Introduce to k-means}
\forceindent K-Means clustering algorithm is proposed by Mac Queen in 1967 which is one of the most popular clustering methods. It is a type of unsupervised learning algorithms that solve the well-known clustering problem. The aim of the algorithm is to define groups of the given data set which called clusters, with the number of clusters represented by the variable K. It iteratively reassigns each data element to one of K clusters based on the provided features and figures cluster centers based on the average of the data locations. Data elements are clustered based on feature similarity.

A simple illustration of a k-means algorithm is considered by using the following data set which comprise 8 two-dimensional coordinates.
\begin{center}
	\begin{tabular}{ | l | l | l |}
		\hline
		Points & X & Y \\ \hline
		1 & 5.1 & 3.5 \\ \hline
		2 & 4.9 & 3.0 \\ \hline
		3 & 4.7 & 3.2 \\ \hline
		4 & 7.0 & 3.2 \\ \hline
		5 & 6.9 & 3.1 \\ \hline
		6 & 6.2 & 2.2 \\ \hline
		7 & 7.7 & 3.8 \\ \hline
		8 & 5.7 & 2.5 \\ \hline
	\end{tabular}
\end{center}

\forceindent Applying k-means algorithm to this data set with two cases K = 2 and K = 3, we have the following respective results.
\begin{enumerate}[+)]
	\item Case K = 2:
	\begin{center}
		\begin{tabular}{ | l | l | l |}
			\hline
			 & cluster 1 & cluster 2 \\ \hline
			Points & 1, 2 , 3, 8 & 4, 5, 6, 7  \\ \hline
			Centroid & (5.1, 3.05) & (6.95, 3.08) \\ \hline
		\end{tabular}
	\end{center}
	\item Case K = 3:
	\begin{center}
		\begin{tabular}{ | l | l | l | l |}
			\hline
			& cluster 1 & cluster 2 & cluster 3 \\ \hline
			Points & 1, 2, 3 & 6, 8 & 4, 5, 7 \\ \hline
			Centroid & (4.9, 3.2) & (6.0, 2.4) & (7.2, 3.4) \\ \hline
		\end{tabular}
	\end{center}
		
\end{enumerate}
\subsection{Why does k-means be used?}


\subsection{K-means algorithm}

\subsection{Advantages and disadvantages}
%lead to the next part

 It is used widely in cluster analysis for that the K-means algorithm has higher efficiency and scalability and converges fast when dealing with large data sets. However it also has many deficiencies: the number of clusters K needs to be initialized, the initial cluster centers are arbitrarily selected, and the algorithm is influenced by the noise points.
\begin{itemize}
	\item Advantages
		\begin{enumerate}[+)]
			\item Fast, robust and easier to understand.
			\item Relatively efficient: O(tknd), where n is number of objects, k is number of clusters, d is number dimensions of each object, and t  is number of iterations. Normally, k, t, d are much smaller than n.
			\item Gives best result when data set are distinct or well separated from each other.
		\end{enumerate}
	
	\item Disadvantages
	
	
\end{itemize}
